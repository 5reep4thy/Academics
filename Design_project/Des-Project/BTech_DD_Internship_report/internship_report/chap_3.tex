% Chapter Template

\chapter{Work Done} % Main chapter title

\label{ChapterX} % Change X to a consecutive number; for referencing this chapter elsewhere, use \ref{ChapterX}

\lhead{Chapter X. \emph{Chapter Title Here}} % Change X to a consecutive number; this is for the header on each page - perhaps a shortened title

%----------------------------------------------------------------------------------------
%	SECTION 1
%----------------------------------------------------------------------------------------
The implementation of the audio processing component takes a relatively high amount of time. If both the video processing component and the audio processing component run the same thread, there would be a drastic drop in frame rates. For the aforementioned reason, the application runs on 2 threads. One for the audio processing and one for the video processing.

\section{Video Processing}
The following packages are used for the video processing.
\begin{description}
\item[$\bullet$] cv2
\item[$\bullet$] cvzone
\item[$\bullet$] threading
\item[$\bullet$] time
\item[$\bullet$] os
\end{description}
\subsection{Video capture}
The frames are obtained from capturing the images from the web camera.
\begin{lstlisting}[breaklines]
cam = cv2.VideoCapture(index) 
#index -> Index of the camera to be used
img = cam.read()
#img -> The frame captured from the camera
\end{lstlisting}

Pseudocode for the video
\begin{lstlisting}[breaklines]
while True:
	img = cam.read()
	display(img)
\end{lstlisting}

\subsection{Background detection and removal}
The background detection and changing it to another image is implemented using the segmentor library of cvzone.
\begin{lstlisting}[breaklines]
imgOutput = segmentor.removeBG(img, ImgWithFlag, threshold)
\end{lstlisting}
The flag, defines the sentiment values, given by the audio detection thread. Based on the value of the imgWithFlag, a suitable image is chosen for the background.

The frame rate is calculated using the cvzone library function.
\begin{lstlisting}[breaklines]
fps = cvzone.FPS()
\end{lstlisting}



\section{Audio processing}
The following packages are used for the audio processing.
\begin{description}
\item[$\bullet$] textBlob
\item[$\bullet$] speech-recognition
\item[$\bullet$] threading
\item[$\bullet$] os
\end{description}
\subsection{Audio capture}
The audio is captured via the microphone using the Recognizer from speech-recognition library.
\begin{lstlisting}[breaklines]
receiver = speech-reconition.Recognizer()
with receiver.Microphone(device_index = 0) as source:
	audio = receiver.listen(source)
\end{lstlisting}

\subsection{Audio to text}
The audio is converted to text using the google speech to text recognizer.
\begin{lstlisting}[breaklines]
text = receiver.recognize_google(audio)
\end{lstlisting}

\subsection{Sentiment analysis on text}
Once the text has been obtained from the audio, we analyse the sentiments with the help of the textBlob library.
\begin{lstlisting}[breaklines]
sentimentObject = TextBlob(text)
polarityOfText = sentimentObject.sentiment.polarity
\end{lstlisting}
\subsection{Give sentiment value to Video Processing thread}
Based on the polarity of text, we change the flag value so that the video processing thread can decide which background to keep.
\begin{lstlisting}[breaklines]
if (polarityOfText > 0):
	flag = 1
else:
	flag = 0
\end{lstlisting}

